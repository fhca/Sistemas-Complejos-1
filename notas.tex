Sistemas complejos 1
notas

día 1:
introducción a mapeos
mapeos lineales
definición de punto fijo
definición de órbita
Tarea1: resumen de capítulos 1-3 del Strogatz

día 2:
mapeos no lineales
mapeos sobre funciones lineales definidas por partes f(x)={ 0<=x<=0.5: 2*x, 0.5<x<=1: 2*x-1}
mapeos sobre funciones derivables sen(x)
logística f(x)=R*x*(1-x)
orbita de la logística
retrato/diagrama fase
Tarea2: Ejercicios del capítulo 5 del Devaney


día 3:
períodos sobre la logística
R > 2
caos, definición preliminar (estabilidad)
Exponente de Lyapunov

16.13 exp lyap https://www.youtube.com/watch?v=VP5q2fVRLss

día 4
Parámetro de control
Diagrama de bifurcación
Bifurcaciones: nodo de montura (x'=r+x^2), 
transcrítica
horquilla super y subcrítica

día 5
[Densidad] X\subset Y, \forall x\in X, \exists y\in Y arbitrariamente cercano a x
[Transitividad] f es transitiva si \forall x,y\in X, \forall\epsilon > 0, \exists z\in X, d(z,x)<\epsilon, \exists n>0\st d(f^{n}(z), y)<\epsilon
Teo. Si \exists x\in X \st orb(x) es densa en X \iff X es transitiva
[sensible a condiciones iniciales] f depende sensiblemente a condiciones iniciales si \exists \delta>0 tal que \forall x\in X y \forall \epsilon>0 \exists y\in X, \exists k>0 \st d(x,y)<\epsilon y d(f^{k}(x), f^{k}(y))\geq \delta
(\delta es independiente de x)
definición de sistema dinámico caótico: 
1. Hay una órbita de f densa
2. f es transitiva
3. f depende sensitivamente de las condiciones iniciales

(1+2 => 3)

30 de agosto:
modelos de sistemas complejos:

2 de septiembre:
ruta al caos a traves de duplicación de período (unimodales)
constante de feigenbaum: tasa con que se duplica el período en func. unimodales

termodinamica:
First law: Energy is conserved. The total amount of energy in the universe is constant. Energy can be transformed from one form to another, such as the transformation of stored body energy to kinetic energy of a pushed car plus the heat generated by this action. However, energy can never be created or destroyed. Thus it is said to be “conserved.”

Second law: Entropy always increases until it reaches a maximum value. The total entropy of a system will always increase until it reaches its maximum possible value; it will never decrease on its own unless an outside agent works to decrease it.

conceptos: (lectura: cap 7 Mitchell, definiendo y midiendo la complejidad )
``Auto-organización''
``información y entropía'': el demonio de Maxwell
entropía: la medida de energía que no puede ser transformada en trabajo adicional, sino que es transformada en calos.
macro y micro estados
Definiendo complejidad:
tamaño, entropía, contenido de información del algoritmo, profundidad {lógica, termodinámica}
